\section{Introduction}

We aim to unify commonly performed analysis in the genetic analysis community
such as heritability estimation, genome-wide association studies (GWAS),
phenotype prediction in a broad sense (e.g., disease risk prediction), variance
decomposition, accounting for population structure and family relatedness, regressing
out instrumental noise, and many others, in the sense that heterogeneous traits
from many different sorts of stochastic nature could be performed in a
non-\textit{ad hoc} manner, with little to no human intervention (we are not
talking about preprocessing here, which obviously will always requires some
considerable human intervention). This clearly could be accomplished by
general-purpose Monte Carlo sampling methods but at the expense of restricting
it to small datasets. Contrarily, we are aiming for at least dozens of thousands
individuals and genome-wide scale numbers of genetic markers which nowadays
easily break under a million points.

The above will be accomplished by the following items:

\begin{enumerate}
  \item Latent variable to model the genetic, environmental, and the covariate parts.
  \item Family of distributions for tackling many sorts of trait nature
  \item Link function connecting the latent variable to the observed outcome
  \item Fast method for integrating out nuisance parameters and to infer the one needed for genetic analysis
  \item Convergence should be guaranteed without the need of human intervention
  \item A model that allows genetic analysis independently of the used outcome distribution
  \item It should be able to handle different traits of stochastic nature jointly
  \item After all, the proposed method has to be empirically as accurate as Monte Carlo sampling methods
\end{enumerate}
