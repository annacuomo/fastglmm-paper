\section{Model}

The underlying unobserved trait is modelled by the random variable
\begin{*align}
z = a + b + \epsilon \tag{1}
\end{*align}
where $a$ is the fixed-effect component (also known as covariate effects), $b$ is the random-effect component, and $\epsilon$ is the i.i.d. normally-distributed noise. The fixed-effect is defined by a dot-product between a vector of covariates and a vector of fixed-effect sizes:
\begin{*align}
a = \mathbf a\T \boldsymbol\alpha
\end{*align}
The random-effect is also defined by a dot-product but now between a vector of stuff and a vector of random-effect sizes:
\begin{*align}
b = \mathbf b\T \bbeta, \qquad \bbeta \sim \mathcal N(0, \sigma^2_{\beta})
\end{*align}

Finally, the i.i.d. noise $\epsilon$ is normally distributed with variance $\sigma^2_{\epsilon}$. If we assume independence and that $\mathrm E[\mathbf b_s] = 0$ and $\mathrm V[\mathbf b_2]=1/\sqrt{n_b}$, we have $\mathrm V[b] = \sigma^2_{\beta}$ denoting the overall effect-size of the random component $b$. If we knew the values of vector $\boldsymbol\alpha$  and made analogous assumptions about the covariates, we would have $\mathrm V[a]  = \sum_{j=1}^{n_a} \boldsymbol \alpha_j^2$. It is under this underlying trait we define genetic concepts as narrow-sense heritability
\begin{*align}
h^2=\frac{\sigma^2_{\beta}}{\sigma_t^2}, \qquad \sigma_t^2 = \sum_{j=1}^{n_a} \boldsymbol \alpha_j^2 + \sigma^2_{\beta} + \sigma^2_{\epsilon}
\end{*align}

or as additive effect-size $\balpha_j$ of a genetic variant $\mathbf a_j$.

In practice however we observe traits that clearly do not follow a Normal distribution, and as such the linear equation expressed in Eq. (1) is considered to describe a process we experimentally don't see. This unobserved process though is assumed to be directly associated with the observed one via a link function and its mean definition:
\begin{*align}
g(\mathrm E[y|z]) = z
\end{*align}
Fig. 1 shows the corresponding Graphical Model:

<img src="model graph.jpg" alt="model graph" style="width: 200px;"/>

Therefore the user is free to choose the distribution that represent the observed process, as long as it can be completly defined by specifying its mean.
